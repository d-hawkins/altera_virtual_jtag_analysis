
\documentclass[10pt,twoside]{article}

% Math symbols
\usepackage{amsmath}
\usepackage{amssymb}

% Headers/Footers
\usepackage{fancyhdr}

% Colors
\usepackage[usenames,dvipsnames]{color}

% Importing and manipulating graphics
\usepackage{graphicx}
\usepackage{subfig}
\usepackage{lscape}

% Misc packages
\usepackage{verbatim}
\usepackage{dcolumn}
\usepackage{ifpdf}
\usepackage{enumerate}

% PDF Bookmarks and hyperref stuff
\usepackage[
  bookmarks=true,
  bookmarksnumbered=true,
  colorlinks=true,
  filecolor=blue,
  linkcolor=blue,
  urlcolor=blue,
  hyperfootnotes=true
  citecolor=blue
]{hyperref}

% Improved citation handling
% (include after the hyperref stuff)
\usepackage{cite}

% Pretty-print code
\usepackage{listings}

% -----------------------------------------------------------------
% Setup the margins
% -----------------------------------------------------------------
% Footer Template

% Set left margin - The default is 1 inch, so the following
% command sets a 1.25-inch left margin.
\setlength{\oddsidemargin}{0.25in}
\setlength{\evensidemargin}{0.25in}

% Set width of the text - What is left will be the right
% margin. In this case, right margin is
% 8.5in - 1.25in - 6in = 1.25in.
\setlength{\textwidth}{6in}

% Set top margin - The default is 1 inch, so the following
% command sets a 0.75-inch top margin.
%\setlength{\topmargin}{-0.25in}

% Set height of the header
\setlength{\headheight}{0.3in}

% Set vertical distance between the header and the text
\setlength{\headsep}{0.2in}

% Set height of the text
\setlength{\textheight}{8.5in}

% Set vertical distance between the text and the
% bottom of footer
\setlength{\footskip}{0.4in}

% -----------------------------------------------------------------
% Allow floats to take up more space on a page.
% -----------------------------------------------------------------

% see page 142 of the Companion for this stuff and the
% documentation for the fancyhdr package
\renewcommand{\textfraction}{0.05}
\renewcommand{\topfraction}{0.95}
\renewcommand{\bottomfraction}{0.95}
% dont make this too small
\renewcommand{\floatpagefraction}{0.35}
\setcounter{totalnumber}{5}

%-----------------------------------------------------------
% Abbreviated symbols
%-----------------------------------------------------------
\newcommand{\sinc}{\ensuremath{\,\text{sinc}}}
\newcommand{\rect}{\ensuremath{\,\text{rect}}}

% =================================================================
% The document starts here
% =================================================================
%
\begin{document}
\title{Altera Virtual JTAG Interface}
\author{D. W. Hawkins (dwh@ovro.caltech.edu)\\
        $$Revision: 1.6 $$}
\date{\today}
%\date{April 25, 2006}
\maketitle

% No header/footer on the first page
\thispagestyle{empty}

\tableofcontents

% start the intro on an odd page
\cleardoublepage
%\clearpage

% Set up the header/footer
\pagestyle{fancy}
\lhead{Altera Virtual JTAG Interface}
\chead{}
\rhead{\today}
\lfoot{}
\cfoot{}
\rfoot{\thepage}
\renewcommand{\headrulewidth}{0.4pt}
\renewcommand{\footrulewidth}{0.4pt}

% Set the listings package language to Tcl
\lstset{language=Tcl}

% =================================================================
\section{Introduction}
% =================================================================

Altera provides a suite of System-Level Debugging (SLD) 
Intellectual Property (IP) cores to aid in FPGA debugging.
These cores provide access to the programmable logic fabric
of the FPGA via the FPGA JTAG interface.
The suite of tools is described in the Quartus II Handbook,
Volume 3, Section 4, {\em System Debugging 
Tools}~\cite{Altera_Quartus_Handbook_2010}.
The suite of tool includes;
%
\begin{itemize}
\item System Console
\item Transceiver Toolkit
\item Signal Tap II Logic Analyzer
\item Signal Probe
\item Logic Analyzer Interface (LAI)
\item In-System Sources and Probes
\item In-System Memory Content Editor
\item Virtual JTAG Interface
\end{itemize}
%
Chapter 13 of the handbook provides an overview of
these tools, while Chapters 14 through 20 describe all but
the Virtual JTAG component in detail.
These debug tools consists of a hardware component, and
Tcl procedures for interfacing to the hardware. 
The hardware consists of user-defined instances of
debug cores, along with a JTAG hub component that 
coordinates access. The presence of the hub component 
is transparent to the user (it is abstracted by the Tcl
procedures).

A typical application of these cores is the
development of new components. For example, consider the
development of a new Altera Avalon bus slave component. 
The component can be tested by creating an SOPC Builder
system design containing an JTAG-to-Avalon Master (or JTAG-to-Avalon 
Streaming) interface and the component.
Transactions to the component can then be generated using
System Console Tcl procedures. A Signal Tap II logic analyzer
instance could be added to the design to capture the
Avalon bus transactions.

The Virtual JTAG Interface (VJI) component provides the
lowest-level JTAG access of all the debug cores. Developing
custom debug components using this interface
provides the maximum flexibility to a user, eg., a user can
define their own JTAG-to-Avalon interfaces and include
features not available in the Altera provided core.
The Virtual JTAG component is described in;
%
\begin{itemize}
\item The Virtual JTAG Megafunction User 
Guide~\cite{Altera_Virtual_JTAG_2008}
\item Quartus II help. From the Contents, see;
\begin{itemize}
\item {\em Using Altera Megafunctions$\rightarrow$Megafunctions/LPM},\\
\rule{5mm}{0pt}{\em JTAG-accessible Extensions MegaWizards and MegaFunctions}
\item {\em Devices and Adapters$\rightarrow$API Functions for Tcl},\\
\rule{5mm}{0pt}\verb+::quartus::jtag 1.0+
\end{itemize}
\end{itemize}
%
The Virtual JTAG User's Guide contains the majority of information
required to use the Virtual JTAG interface. However, the Altera
documentation has errors, and the simulation models are not useful
for automated component testing.

This document provides an analysis of the Altera Virtual JTAG
documentation, simulation, and synthesis results. It provides a
clear description of how the interface works, and provides a
simulation model that can be used for automated component testing.
A general-purpose Virtual JTAG component is developed that allows
custom device identification. Example designs are provided.

\clearpage
% =================================================================
\section{Virtual JTAG Component Analysis}
% =================================================================

This section analyzes Altera Virtual JTAG component instantiation,
simulation, and synthesis. The section starts by describing
Virtual JTAG component instantiation. The simulation results
for that component show numerous problems with the Altera simulation
model. Tcl procedures for performing JTAG accesses are then
described, and used to generate hardware transaction sequences.
The hardware transaction sequences are captured using an external 
logic analyzer and a SignalTap II logic analyzer. The traces are
used to show the problems with the Altera simulation model, and 
are used in Section~\ref{sec:simulation_model} to develop a 
functionally correct simulation model.
The section ends with a description of the Altera hub and
node instruction encoding, and the effect on that encoding when 
designs contain multiple SLD components.

% -----------------------------------------------------------------
\subsection{Component Instantiation}
% -----------------------------------------------------------------
\label{sec:sld_virtual_jtag_instantiation}

The Virtual JTAG core can be instantiated using the Quartus II
MegaWizard ({\em Tools$\rightarrow$MegaWizard Plug-In Manager})
or by direct instantiation (see the Quartus II help for
the template). 
%
An Altera DE2 board Virtual JTAG test configuration was
created as follows;
%
\begin{itemize}
\item Using the procedure described in Appendix~\ref{app:source_code},
a new project for the DE2 board was created. The project folder
was named \verb+sld_vjtag_to_gpio+, with top-level component
\verb+de2.vhd+.

The \verb+sld_vjtag_to_gpio+ top-level design, \verb+de2.vhd+, contains an
\verb+sld_virtual_jtag+ instance connected to the DE2 board LEDs,
hexadecimal displays, switches, and the GPIO (for external
logic analyzer transaction capture). This section
describes how the Virtual JTAG instance was instantiated;
see the source code for details on the connections to
the DE2 board I/O.

\item A Virtual JTAG instance was created as follows;
\begin{enumerate}
\item A new Quartus project was created in a temporary work folder
(the project targeted the Cyclone II on the DE2 board).
\item The MegaWizard was used to create a Virtual JTAG instance
named \verb+vji+, with the settings shown in 
Figure~\ref{fig:sld_virtual_jtag_megawizard_pg3}, i.e.,
\begin{itemize}
\item Virtual instruction register width of 3.
\item Enable the primitive JTAG state signal ports.
\end{itemize}
\item No other MegaWizard defaults were changed.
\item Click `Finish' to have the \verb+vji.vhd+ instance file
created.
\end{enumerate}
%
\item The \verb+sld_virtual_jtag+ instance from \verb+vji.vhd+
was then copied-and-pasted into \verb+de2.vhd+
(do not delete the Quartus work folder, as the MegaWizard
will be used to create simulation stimulus in the next section). 
\item The library definition for the \verb+sld_virtual_jtag+
component was added to \verb+de2.vhd+, i.e., 
%
\begin{verbatim}
library altera_mf;
use altera_mf.altera_mf_components.all;
\end{verbatim}
%
\item Internal signals were attached to all of the \verb+sld_virtual_jtag+
ports, and those signals were connected to the \verb+gpio_a+ and
\verb+gpio_b+ ports (the DE2 board 2$\times$20 100-mil headers),
so that the signals can be probed with an external logic
analyzer. (See the \verb+de2.vhd+ source for the JTAG signal to GPIO
assignments).
\item The Virtual instruction register bus from the Virtual
JTAG component was connected to GPIO bits and to the LEDs
(so that Tcl commands can change the LED state).
\item The Virtual instruction register bus to the Virtual
JTAG component was connected to the DE2 switches
(so that Tcl commands can read the switch state).
\item The JTAG TDO signal was configured to toggle on
the rising-edge of the JTAG TCK signal.
\item The design was synthesized.
\end{itemize}

\begin{figure}[t]
  \centering
  \includegraphics[width=0.8\textwidth]
  {figures/sld_virtual_jtag_megawizard_pg3.jpg}
  \caption{Altera Virtual JTAG component configuration.}
  \label{fig:sld_virtual_jtag_megawizard_pg3}
\end{figure}

\clearpage
% -----------------------------------------------------------------
\subsection{Altera Simulation Model Analysis}
% -----------------------------------------------------------------
\label{sec:sld_virtual_jtag_simulation}

\begin{figure}[t]
  \centering
  \includegraphics[width=0.8\textwidth]
  {figures/sld_virtual_jtag_megawizard_pg4.jpg}
  \caption{Altera Virtual JTAG simulation stimulus configuration.}
  \label{fig:sld_virtual_jtag_megawizard_pg4}
\end{figure}

The Virtual JTAG component has simulation support, however, the
waveforms the simulation model generates are \textcolor{red}{incorrect}, making
the simulation feature useless. Even if simulation worked, the
transaction support is so basic, that it would
be barely useful; the model supports the generation of
statically-defined, write-only, Virtual IR-shift and Virtual
DR-shift operations using 
generics\footnote{The stimulus could conceivably be changed during
run-time using multiple runs of Modelsim's {\tt vsim} with
different generics each run, however, this is not a standard
way to construct a self-verifying testbench.}. The fact that the
stimulus is write-only makes writing a self-verifying testbench 
impossible, as there is no read-back method.
%
This section describes how to create stimulus and contains figures
showing the simulation IR-shift and DR-shift operations. The
simulation waveforms are annotated to show the model problems, 
relative to the hardware measurements
performed in the next section.

Figure~\ref{fig:sld_virtual_jtag_megawizard_pg4} shows the MegaWizard
simulation stimulus page; use the MegaWizard to edit the previously
generated instance of the Virtual JTAG component and enter the
values as shown in the figure, then regenerate the VHDL files
(by clicking `Finish'). The generics in the original
instance (copied to the \verb+de2.vhd+ file) were;
%
\begin{verbatim}
sld_sim_action          => "",
sld_sim_n_scan          => 0,
sld_sim_total_length    => 0,
\end{verbatim}
%
The addition of the stimulus changes the \verb+vji.vhd+
generics to;
%
\begin{verbatim}
sld_sim_action => "((1,1,1,3),(1,1,2,3),(1,1,3,3),(1,2,4,8),(1,2,5,8),(1,2,6,8))",
sld_sim_n_scan => 6,
sld_sim_total_length => 33,
\end{verbatim}
%
The stimulus format is; delay between transactions (in units of 1ms),
operation (1 = IR-shift, 2 = DR-shift), IR/DR value, and IR/DR bit-width.

\clearpage
The Virtual JTAG instance created in \verb+vji.vhd+ contains the stimulus
via the generics, so the design can be simulated directly without
wrapping the component in a testbench (the input ports \verb+tdo+
and \verb+ir_in[2:0]+ ports will be undriven so the Modelsim simulation
will show these signals as unknown/undriven). Figures~\ref{fig:vji_reset_tb} 
through~\ref{fig:vji_shift_dr_tb_b} show time segments from
the simulation. The figures show; the JTAG waveforms, the JTAG TAP one-hot
state machine encoded signals, the Virtual JTAG one-hot state
machine encoded signals, and the Virtual Instruction Register 
parallel input/output ports. Comments on the start-up and
Virtual IR-shift sequences are;
%
\begin{itemize}
%
\item Figure~\ref{fig:vji_reset_tb} shows the simulation waveforms 
after power-on. The waveforms show a JTAG TAP reset sequence 
(\verb+tms+ high for six \verb+tck+ rising-edges), followed
by a JTAG TAP instruction phase, containing 10-bits of zeros.
There is no Altera device 10-bit JTAG instruction code of
\verb+00_0000_0000b = 000h+, so this sequence is not useful.

The stimulus leaves the JTAG TAP state-machine in the Update-IR state
(the one-hot JTAG state \verb+jtag_state_uir+ remains asserted
at the end of the simulation segment).
%
\item Figure~\ref{fig:vji_shift_ir_tb_a} shows the simulation waveforms
for the first Virtual IR-shift stimulus, where the stimulus value for the
IR shift was 1. The stimulus should consist of three JTAG TAP
shift sequences;
%
\begin{itemize}
\item JTAG TAP IR-shift of 10-bits \verb+USER1 = 00_0000_1110b = 00Eh+
\item JTAG TAP DR-shift of 5-bits \verb+VIR_CAPTURE = 0_1011b = 0Bh+
\item JTAG TAP DR-shift of 5-bits \verb+VIR_USER = 1_0001b = 11h+
\end{itemize}
%
where the last two sequences are 5-bit Virtual IR-shift codes
(Section~\ref{sec:vir_encoding} discusses the bit
encoding of the Virtual IR-shift codes).
The three JTAG TAP shift sequences are present in the simulation,
however, the TDI activity is incorrect, and the last DR-shift
phase (the Virtual IR-shift user instruction phase, \verb+VIR_USER+)
is missing a shift phase.

The stimulus leaves the JTAG TAP state-machine in the Update-DR state
and the Virtual JTAG state-machine in the Update-IR state
(the one-hot JTAG state \verb+jtag_state_udr+ and Virtual JTAG
state \verb+vjtag_state_uir+ remain asserted at the end of the
simulation segment).

The Virtual JTAG instruction register updates to \verb+1+ when
the Virtual JTAG state machine enters the Update-IR state.
%
\item Figure~\ref{fig:vji_shift_ir_tb_b} shows the simulation waveforms
for the second Virtual IR-shift stimulus, where the stimulus value for the
IR shift was 2. Since the JTAG TAP instruction \verb+USER1+ was shifted
in during the first Virtual IR-shift stimulus, this instruction phase
is not required for the second stimulus, so the second stimulus waveforms
consists of the last two shift sequences seen in the
waveforms of the first stimulus, i.e., \verb+VIR_CAPTURE+ followed by 
\verb+VIR_USER+. As with the first Virtual IR-shift sequence,
the \verb+VIR_USER+ sequence is missing a shift phase.

The Virtual JTAG instruction register updates to \verb+2+ when
the Virtual JTAG state machine enters the Update-IR state.
%
\item The third Virtual IR-shift stimulus\footnote{Not shown in this
document. Run the Modelsim simulation to see it.} results in 
waveforms similar to the second Virtual IR-shift 
stimulus, with the Virtual JTAG instruction register updating
to \verb+3+ at the end of the stimulus sequence
(prior to the Virtual DR-shift stimulus).
%
\end{itemize}

\begin{figure}[t]
  \centering
  \includegraphics[width=\textwidth]
  {figures/vji_reset_tb.pdf}
  \caption{Altera Virtual JTAG simulation; power-on sequence.}
  \label{fig:vji_reset_tb}
\end{figure}

\begin{figure}[t]
  \centering
  \includegraphics[width=\textwidth]
  {figures/vji_shift_ir_tb_a.pdf}
  \caption{Altera Virtual JTAG simulation; first IR-shift sequence
(stimulus value of 1).}
  \label{fig:vji_shift_ir_tb_a}
\end{figure}

\begin{figure}[t]
  \centering
  \includegraphics[width=\textwidth]
  {figures/vji_shift_ir_tb_b.pdf}
  \caption{Altera Virtual JTAG simulation; second IR-shift sequence
(stimulus value of 2).}
  \label{fig:vji_shift_ir_tb_b}
\end{figure}

\clearpage
\noindent Comments on the Virtual DR-shift stimulus are;
\begin{itemize}
\item Figure~\ref{fig:vji_shift_dr_tb_a} shows the simulation waveforms
for the first Virtual DR-shift stimulus, where the stimulus value for the
8-bit Virtual DR-shift was 4. The stimulus should consists of two JTAG TAP
shift sequences;
%
\begin{itemize}
\item JTAG TAP IR-shift of 10-bits \verb+USER0 = 00_0000_1100b = 00Ch+
\item JTAG TAP DR-shift of 8-bits \verb+VDR_USER = 0000_0100b = 04h+
\end{itemize}
%
The two JTAG TAP shift sequences are present in the simulation,
however, there are a couple of problems; there is an extra shift
clock during the instruction phase, and the TDI activity is incorrect.
The TDI signal has no activity during the instruction phase, and 
during the DR-shift phase, \verb+tdi+ asserts during the wrong bit
position (based on the other DR-shift stimulus, it appears that the
DR-shift value is being shifted out one clock late).

The stimulus leaves the JTAG TAP and Virtual JTAG state-machines
in the Update-DR state
(the one-hot states \verb+jtag_state_udr+ and \verb+vjtag_state_udr+ 
remain asserted at the end of the
simulation segment).
%
\item Figure~\ref{fig:vji_shift_dr_tb_b} shows the simulation waveforms
for the second Virtual DR-shift stimulus, where the stimulus value for the
8-bit Virtual DR-shift was 5. 
Since the JTAG TAP instruction \verb+USER0+ was shifted
in during the first Virtual DR-shift stimulus, this instruction phase
is not required for the second stimulus, so the second stimulus waveforms
consists of the last shift sequence seen in the
waveforms of the first stimulus, i.e., \verb+VDR_USER+. 
As with the first Virtual DR-shift stimulus,
the Virtual DR-shift value, \verb+VDR_USER = 05h+, is shifted
out one clock late.
%
\item The third Virtual DR-shift stimulus\footnote{Not shown in this
document. Run the Modelsim simulation to see it.} results in 
waveforms similar to the second Virtual DR-shift 
stimulus, with the Virtual DR-shift value, \verb+VDR_USER = 06h+,
shifted out one clock late
%
\item The simulation ends with the JTAG TAP being reset
(\verb+tms+ high for six \verb+tck+ rising-edges).
%
\end{itemize}

\begin{figure}[t]
  \centering
  \includegraphics[width=\textwidth]
  {figures/vji_shift_dr_tb_a.pdf}
  \caption{Altera Virtual JTAG simulation; first DR-shift sequence
(stimulus value of 4).}
  \label{fig:vji_shift_dr_tb_a}
\end{figure}

\begin{figure}[t]
  \centering
  \includegraphics[width=\textwidth]
  {figures/vji_shift_dr_tb_b.pdf}
  \caption{Altera Virtual JTAG simulation; second DR-shift sequence
(stimulus value of 5).}
  \label{fig:vji_shift_dr_tb_b}
\end{figure}

\clearpage
% -----------------------------------------------------------------
\subsection{Altera Tcl procedures for JTAG access}
% -----------------------------------------------------------------
\label{sec:tcl_procedures}

Virtual JTAG communications are performed using procedures
from the \verb+::quartus::jtag 1.0+ Tcl package. 
Table~\ref{tab:quartus_jtag} summarizes the Tcl procedures available
in the package. Detailed information on the Tcl procedures are 
provided in the Quartus II help 
({\em Contents$\rightarrow$Devices and Adapters$\rightarrow$API Functions for Tcl}),
in the \verb+quartus_stp+ Tcl console (use \verb+-help+ or
\verb+-long_help+ as an option to the Tcl procedure), and
in the Quartus II scripting manual~\cite{Altera_Tcl_Manual_2009}.
Figure~\ref{fig:tcl_open_close} shows the open/close sequence
for a single USB controller connected to a single device JTAG chain,
eg., the Altera DE2 board with embedded USB-Blaster and Cyclone II 
device.
The Quartus help recommends holding the JTAG lock for the minimum
time possible to allow other applications to access the device.

The JTAG IR-shift and DR-shift procedures require some 
caution with their use, the following subtleties need to be observed;
%
\begin{itemize}
\item The \verb+device_ir_shift+ and \verb+device_virtual_ir_shift+
procedure \verb+-ir_value+ argument is a Tcl numeric value,
eg., the JTAG TAP \verb+USER1+ instruction can be issued as
decimal 14, hexadecimal 0xE, or octal 016
(Tcl commands \verb+expr 0xE+ and \verb+expr 0x16+ both return 14).
%
\item The \verb+device_dr_shift+ and \verb+device_virtual_dr_shift+
procedure \verb+-dr_value+ argument is a binary string by default.
When the \verb+-value_in_hex+ option is used, the 
\verb+-dr_value+ argument is a hexadecimal string {\em without}
a leading \verb+0x+ prefix. The length of the binary or hexadecimal string
must be sufficient to match the number of bits indicated by
the \verb+-length+ argument.
%
\item If a JTAG or Virtual JTAG shift command is issued with 
the option \verb+-no_captured_ir_value+ or \verb+-no_captured_dr_value+,
{\em no activity} will occur on the JTAG interface until a
command without this option or a \verb+device_unlock+ command
is issued.
\end{itemize}
%
An additional subtlety comes from Tcl; 
Figure~\ref{fig:tcl_integer_support} demonstrates that Tcl supports 
32-bit signed integers. Arguments to the Tcl procedures for JTAG and 
Virtual JTAG access reflect this limitation; \verb+-ir_value+ is an integer,
whereas \verb+-dr_value+ is either a binary or hexadecimal string
(allowing for larger than 32-bit values).
Since the argument to a Virtual IR-shift procedure is limited to
32-bits, the \verb+sld_virtual_jtag+ component IR width, \verb+SLD_IR_WIDTH+,
is also limited to 32-bits. At the hardware level, a 32-bit
Virtual IR-shift instruction has address bits prepended to
the instruction, and is issued as a JTAG DR-shift instruction
(see Section~\ref{sec:vir_encoding} for details on the instruction encoding).

The sequence of JTAG TAP transactions that are issued for
a Virtual JTAG transaction depends on the previously issued
transactions; this can be seen in the simulations in
Section~\ref{sec:sld_virtual_jtag_simulation}, and in the 
logic analyzer traces in Sections~\ref{sec:external_logic_analyzer_traces}
and~\ref{sec:signaltap_logic_analyzer_traces}.
The sequence of JTAG TAP transactions that are issued for
a Virtual JTAG transaction can be reported by the 
Virtual JTAG Tcl procedures by adding the option
\verb+-show_equivalent_device_ir_dr_shift+.
Figures~\ref{fig:tcl_shift_ir} and~\ref{fig:tcl_shift_dr}
show the Virtual IR-shift and DR-shift sequences that result
in different JTAG transactions. These commands are used
in the next sections to generate JTAG transactions that
are captured via logic analyzer.

\clearpage
\begin{table}[p]
\caption{Quartus JTAG ({\tt ::quartus::jtag 1.0}) Tcl procedures.}
\label{tab:quartus_jtag}
\begin{center}
\begin{tabular}{|l|p{9cm}|}
\hline
Command & Description\\
\hline\hline
&\\
\verb+get_hardware_names+ & List the JTAG hardware, eg. USB-Blaster.\\
&\\
\verb+get_device_names+ & List the devices on a JTAG bus.\\
&\\
\verb+open_device+ & Access a device on a JTAG bus.\\
\verb+close_device+ & End access to a device.\\
&\\
\verb+device_lock+   & Lock the JTAG interface while accessing a device.\\
\verb+device_unlock+ & Release the JTAG interface lock.\\
&\\
\verb+device_run_test_idle+ & Move the JTAG TAP to run-test/idle.\\
\verb+device_dr_shift+      & Issue a JTAG TAP DR-shift sequence.\\
\verb+device_ir_shift+      & Issue a JTAG TAP IR-shift sequence.\\
&\\
\verb+device_virtual_dr_shift+ & Issue a Virtual JTAG DR-shift sequence.\\
\verb+device_virtual_ir_shift+ & Issue a Virtual JTAG IR-shift sequence.\\
&\\
\hline
\end{tabular}
\end{center}
\end{table}

% listings package
%   The pretty-printing was not formatting the comments correctly.
%   The word 'first' was getting highlighted as a keyword.
%   For now, just use verbatim. Actually, verbatim did not work
%   with the framebox, so just use \tt inside the minipage.
%   Grr, but then the hash and underscores need to be escaped.
%   Ok, skip the framebox around this one then!
%
\begin{figure}[p]
\begin{center}
\begin{minipage}{100mm}
%\begin{lstlisting}
\begin{verbatim}
proc jtag_open {} {
    # Get the list of JTAG controllers
    set hardware_names [get_hardware_names]

    # Select the first JTAG controller
    set hardware_name [lindex $hardware_names 0]

    # Get the list of FPGAs in the JTAG chain
    set device_names [get_device_names\
        -hardware_name $hardware_name]

    # Select the first FPGA
    set device_name [lindex $device_names 0]

    puts "\nJTAG: $hardware_name, FPGA: $device_name"
    open_device -hardware_name $hardware_name\
        -device_name $device_name
}

proc jtag_close {} {
    close_device
}
\end{verbatim}
%\end{lstlisting}
\end{minipage}
\end{center}
\caption{Tcl procedures for open and close.}
\label{fig:tcl_open_close}
\end{figure}

\begin{figure}[t]
\begin{center}
\begin{minipage}{100mm}
%\begin{lstlisting}
\begin{verbatim}
proc print_ints {width} {
    for {set i 0} {$i < $width} {incr i} {
        set val [expr {1<<$i}];
        puts "[format "%2d: %9X %12d" $i $val $val]"
    }
}
\end{verbatim}
%\end{lstlisting}
\end{minipage}

(a)

\vskip5mm
\begin{minipage}{60mm}
\begin{verbatim}
tcl> print_ints 35
 0:         1            1
 1:         2            2
 2:         4            4
 3:         8            8
 4:        10           16
 5:        20           32
 6:        40           64
 7:        80          128
 8:       100          256
 9:       200          512
10:       400         1024
11:       800         2048
12:      1000         4096
13:      2000         8192
14:      4000        16384
15:      8000        32768
16:     10000        65536
17:     20000       131072
18:     40000       262144
19:     80000       524288
20:    100000      1048576
21:    200000      2097152
22:    400000      4194304
23:    800000      8388608
24:   1000000     16777216
25:   2000000     33554432
26:   4000000     67108864
27:   8000000    134217728
28:  10000000    268435456
29:  20000000    536870912
30:  40000000   1073741824
31:  80000000  -2147483648
32:         0            0
33:         0            0
34:         0            0
tcl>
\end{verbatim}
\end{minipage}

(b)
\end{center}
\caption{Tcl 32-bit integer support; (a) Tcl procedure
{\tt print\_ints} to print integer values with widths
of 1 to {\tt width} bits, and (b) conversion values for 
widths of 1 to 35-bits. The output values indicate
that Tcl supports signed 32-bit integers.}
\label{fig:tcl_integer_support}
\end{figure}

% quartus_stp Tcl session
%
% Note: to get the info text formatted in green, a minipage is used
% along with the color package. The font is changed to typewriter
% style, and the info text is changed to green to match the 
% appearance in quartus_stp. The green text is indented using
% rule. The indenting could be implemented with hspace, but blank
% lines would be needed between lines (so that each line was
% interpreted as a new paragraph).
%
\begin{figure}[p]
\hfil
\framebox{
\hspace{1mm}
\begin{minipage}{130mm}
\vspace{1mm}
\tt
\color{black}
tcl> jtag\_open\\
JTAG: USB-Blaster [USB-0], FPGA: @1: EP2C35 (0x020B40DD)\\
tcl> device\_lock -timeout 10000\\
tcl> device\_virtual\_ir\_shift -instance\_index 0 -ir\_value 1 
-show\_equivalent\_device\_ir\_dr\_shift\\
\color{green}
Info: Equivalent device ir and dr shift commands\\
\rule{5mm}{0pt} Info: device\_ir\_shift -ir\_value 14\\
\rule{5mm}{0pt} Info: device\_dr\_shift -length 5 -dr\_value 0B -value\_in\_hex\\
\rule{5mm}{0pt} Info: device\_dr\_shift -length 5 -dr\_value 11 -value\_in\_hex\\
\color{black}
5\\
tcl> device\_virtual\_ir\_shift -instance\_index 0 -ir\_value 2 
-show\_equivalent\_device\_ir\_dr\_shift\\
\color{green}
Info: Equivalent device ir and dr shift commands\\
\rule{5mm}{0pt} Info: device\_dr\_shift -length 5 -dr\_value 0B -value\_in\_hex\\
\rule{5mm}{0pt} Info: device\_dr\_shift -length 5 -dr\_value 12 -value\_in\_hex\\
\color{black}
5\\
tcl> device\_virtual\_ir\_shift -instance\_index 0 -ir\_value 3 
-show\_equivalent\_device\_ir\_dr\_shift -no\_captured\_ir\_value\\
\color{green}
Info: Equivalent device ir and dr shift commands\\
\rule{5mm}{0pt} Info: device\_dr\_shift -length 5 -dr\_value 12 -value\_in\_hex\\
\color{black}
tcl> device\_unlock\\
tcl> jtag\_close\\
tcl>
\vspace{1mm}
\end{minipage}
}\hfil
\caption{Virtual IR-shift Tcl procedures (using {\tt quartus\_stp}).}
\label{fig:tcl_shift_ir}
%\end{figure}
%
\vskip5mm
%
%\begin{figure}[p]
\hfil
\framebox{
\hspace{1mm}
\begin{minipage}{130mm}
\vspace{1mm}
\tt
\color{black}
tcl> jtag\_open\\
JTAG: USB-Blaster [USB-0], FPGA: @1: EP2C35 (0x020B40DD)\\
tcl> device\_lock -timeout 10000\\
tcl> device\_virtual\_dr\_shift -instance\_index 0 -length 8 -dr\_value 04 -value\_in\_hex  
-show\_equivalent\_device\_ir\_dr\_shift\\
\color{green}
Info: Equivalent device ir and dr shift commands\\
\rule{5mm}{0pt} Info: device\_ir\_shift -ir\_value 12\\
\rule{5mm}{0pt} Info: device\_dr\_shift -length 8 -dr\_value 04 -value\_in\_hex\\
\color{black}
AA\\
tcl> device\_virtual\_dr\_shift -instance\_index 0 -length 8 -dr\_value 05 -value\_in\_hex  
-show\_equivalent\_device\_ir\_dr\_shift\\
\color{green}
Info: Equivalent device ir and dr shift commands\\
\rule{5mm}{0pt} Info: device\_dr\_shift -length 8 -dr\_value 05 -value\_in\_hex\\
\color{black}
AA\\
tcl> device\_virtual\_dr\_shift -instance\_index 0 -length 8 -dr\_value 05 -value\_in\_hex  
-show\_equivalent\_device\_ir\_dr\_shift -no\_captured\_dr\_value\\
\color{green}
Info: Equivalent device ir and dr shift commands\\
\rule{5mm}{0pt} Info: device\_dr\_shift -length 8 -dr\_value 06 -value\_in\_hex\\
\color{black}
tcl> device\_unlock\\
tcl> jtag\_close\\
tcl>
\vspace{1mm}
\end{minipage}
}\hfil
\caption{Virtual DR-shift Tcl procedures (using {\tt quartus\_stp}).}
\label{fig:tcl_shift_dr}
\end{figure}

\clearpage
% -----------------------------------------------------------------
\subsection{External logic analyzer traces}
% -----------------------------------------------------------------
\label{sec:external_logic_analyzer_traces}

Virtual JTAG communications are performed using  Tcl
procedures from the \verb+::quartus::jtag 1.0+ Tcl package 
issued from the \verb+quartus_stp+ application. 
Section~\ref{sec:tcl_procedures} discusses the Tcl
procedures. 
Figures~\ref{fig:tcl_shift_ir} and~\ref{fig:tcl_shift_dr}
show the Virtual IR-shift and DR-shift sequences that result
in different JTAG transactions.
Figures~\ref{fig:vji_shift_ir_hw_a} through~\ref{fig:vji_shift_dr_hw_b} 
logic analyzer traces for the IR-shift and DR-shift sequences
captured from the Altera DE2 board\footnote{The {\tt TDO} signal
was configured to toggle at each {\tt TCK} rising-edge.}.
The logic analyzer sequences are;
%
\begin{itemize}
\item Figure~\ref{fig:vji_shift_ir_hw_a} shows an initial Virtual IR-shift
sequence (after accessing the device), with a user instruction
value of 1, i.e., the first \verb+device_virtual_ir_shift+ command in
Figure~\ref{fig:tcl_shift_ir}.

The captured sequence is slightly different than the sequence expected.
There is an erroneously formed JTAG \verb+BYPASS+ instruction 
issued prior to the JTAG \verb+USER1+ instruction. Alternatively, this
sequence could be a JTAG instruction register length detection
sequence, eg., a single zero followed by ones is shifted into the
JTAG IR register and the control software detects when the
zero appears at the TDO output (the \verb+BYPASS+ command
just happens to be the all-ones instruction). The other minor 
issue with the captured sequence is that the \verb+TDI+
signal shows a glitch during the \verb+USER1+ instruction sequence.
%
\item Figure~\ref{fig:vji_shift_ir_hw_b} shows a Virtual IR-shift
sequence, with an instruction value of 2, following a Virtual IR-shift 
sequence i.e., the second \verb+device_virtual_ir_shift+ command in
Figure~\ref{fig:tcl_shift_ir}.
%
\item Figure~\ref{fig:vji_shift_ir_hw_c} shows a Virtual IR-shift 
sequence, with an instruction value of 3 and the option \newline
\verb+-no_captured_ir_value+
following a Virtual IR-shift sequence, i.e., the third\newline
\verb+device_virtual_ir_shift+ command in
Figure~\ref{fig:tcl_shift_ir}.
%
\item Figure~\ref{fig:vji_shift_dr_hw_a} shows the first Virtual 
DR-shift sequence, with a DR value of 0xAA, i.e., the first\newline
\verb+device_virtual_dr_shift+ command in
Figure~\ref{fig:tcl_shift_dr} (with the data value changed
to 0xAA so that the TDI signal would toggle each data bit).

During the JTAG \verb+USER0+ instruction sequence, the \verb+TDI+
signal shows a glitch (similar to the glitch seen during
the Virtual IR-shift \verb+USER1+ instruction sequence).

The traces show a logical \textcolor{red}{error} in the hardware;
there are two Virtual JTAG one-hot state assertions, i.e.,
both capture-DR (\verb+VS_CDR+) and capture-IR (\verb+VS_CIR+)
assert during the same clock period. 
%
\item Figure~\ref{fig:vji_shift_dr_hw_b} second Virtual DR-shift sequence,
with a DR value of 0xAA, i.e., the second
\verb+device_virtual_dr_shift+ command in
Figure~\ref{fig:tcl_shift_dr} (with the data value changed
to 0xAA so that the TDI signal would toggle each data bit).
\end{itemize}
%
The logic analyzer captures show that the JTAG sequences issued
by each Tcl procedure are consistent with the commands reported
by the option \verb+-show_equivalent_device_ir_dr_shift+.
The logic analyzer traces also provide the state of the
JTAG TAP and Virtual JTAG state-machines after each command is 
issued (the states observed are consistent with the simulation
states). The logic analyzer waveforms confirm the
simulation model errors identified in 
Section~\ref{sec:sld_virtual_jtag_simulation}.

% -----------------------------------------------------------------
\subsection{SignalTap II logic analyzer traces}
% -----------------------------------------------------------------
\label{sec:signaltap_logic_analyzer_traces}

Figures~\ref{fig:vji_shift_ir_stp_a} and~\ref{fig:vji_shift_dr_stp_a}
show Virtual IR-shift and Virtual DR-shift sequences captured
using a SignalTap II logic analyzer instance in the DE2 design.
The traces are very similar to those captured using the external
logic anayzer. The slight difference, is that the presence of the
SignalTap II instances changes the Virtual Instruction encoding
as discussed in Section~\ref{sec:vir_encoding}. The captured
traces show the TDI glitch, and erroneous one-hot state assertion
observed with the external logic analyzer.

\clearpage
\begin{figure}[t]
  \centering
  \includegraphics[width=\textwidth]
  {figures/vji_shift_ir_hw_a.pdf}
  \caption{Altera Virtual JTAG logic analyzer trace; initial Virtual IR-shift
  sequence (after accessing the device). The sequence consists of
  JTAG {\tt BYPASS} and {\tt USER1} IR-shift sequences followed by
  the JTAG DR-shift sequences.}
  \label{fig:vji_shift_ir_hw_a}
\end{figure}

\begin{figure}[t]
  \centering
  \includegraphics[width=\textwidth]
  {figures/vji_shift_ir_hw_b.pdf}
  \caption{Altera Virtual JTAG logic analyzer trace; Virtual IR-shift
  sequence following a Virtual IR-shift sequence (the JTAG {\tt USER1}
  sequence is not issued).}
  \label{fig:vji_shift_ir_hw_b}
\end{figure}

\begin{figure}[t]
  \centering
  \includegraphics[width=\textwidth]
  {figures/vji_shift_ir_hw_c.pdf}
  \caption{Altera Virtual JTAG logic analyzer trace; Virtual IR-shift
  sequence with the option {\tt -no\_captured\_ir\_value} following a Virtual
  IR-shift sequence (the JTAG {\tt USER1} and {\tt VIR\_CAPTURE} sequences
  are not issued).}
  \label{fig:vji_shift_ir_hw_c}
\end{figure}

\begin{figure}[t]
  \centering
  \includegraphics[width=\textwidth]
  {figures/vji_shift_dr_hw_a.pdf}
  \caption{Altera Virtual JTAG logic analyzer trace; Virtual DR-shift
  sequence following a IR-shift sequence.  The sequence consists of
  JTAG {\tt USER0} IR-shift sequence followed by
  a JTAG DR-shift sequence. Note the invalid pulse on {\tt virtual\_state\_cir};
  this indicates a logical error in the Altera component.}
  \label{fig:vji_shift_dr_hw_a}
\end{figure}

\begin{figure}[t]
  \centering
  \includegraphics[width=\textwidth]
  {figures/vji_shift_dr_hw_b.pdf}
  \caption{Altera Virtual JTAG logic analyzer trace; Virtual DR-shift
  sequence following a Virtual DR-shift sequence (the JTAG {\tt USER0} sequence
  is not issued).  Note the invalid pulse on {\tt virtual\_state\_cir};
  this indicates a logical error in the Altera component.}
  \label{fig:vji_shift_dr_hw_b}
\end{figure}

% -----------------------------------------------------------------
% SignalTap II logic analyzer traces
% -----------------------------------------------------------------
%
% The bitmaps generated by SignalTap can be edited using MATLAB to
% eliminate sections of the bitmap. Eg.
%
%    x = imread('de2a.bmp');
%    image(x)
%    y = [x(:,1:265,:) x(:,600:end,:)];
%    image(y)
%    imwrite(y,'de2a.bmp','BMP');
%
% Then take that image into Inkscape for annotation. Don't both to
% scale the bitmap in Inkscape, just leave it in its natural size
% and use Latex to scale the image.
%
\begin{landscape}
\begin{figure}
  \centering
  \includegraphics[width=210mm]
  {figures/vji_shift_ir_stp_a.pdf}
  \caption{Altera Virtual JTAG SignalTap II logic analyzer trace; initial
  Virtual IR-shift sequence. The sequence consists of the JTAG {\tt USER1} 
  IR-shift sequence followed by the JTAG DR-shift sequences.}
  \label{fig:vji_shift_ir_stp_a}
\end{figure}
\end{landscape}

\begin{landscape}
\begin{figure}
  \centering
  \includegraphics[width=210mm]
  {figures/vji_shift_dr_stp_a.pdf}
  \caption{Altera Virtual JTAG SignalTap II logic analyzer trace; Virtual DR-shift
  sequence following a IR-shift sequence.  The sequence consists of the
  JTAG {\tt USER0} IR-shift sequence followed by a JTAG DR-shift sequence. 
  Note the invalid pulse on {\tt virtual\_state\_cir};
  this indicates a logical error in the Altera component.}
  \label{fig:vji_shift_dr_stp_a}
\end{figure}
\end{landscape}

\clearpage
% -----------------------------------------------------------------
\subsection{JTAG hub/node identification and instruction encoding}
% -----------------------------------------------------------------
\label{sec:vir_encoding}

The Appendices in the Virtual JTAG Megafunction User
Guide~\cite{Altera_Virtual_JTAG_2008} provide low-level
details on the JTAG hub and device node identification
and instruction encoding. This section provides details
on how to use these instructions, and provides a correction
relative to the user guide.

The Altera JTAG debug devices are coordinated via the JTAG
hub (reported in the Quartus hierarchy window as the 
\verb+sld_hub+). The properties of the hub and the nodes
attached to the hub can be queried using the Virtual
Instruction Register (VIR) \verb+HUB_INFO+ instruction.
Figure~\ref{fig:vir_encoding} shows the VIR instruction
encoding, while Figures~\ref{fig:hub_info} 
and~\ref{fig:node_info} show the hub and node information
that is accessed using the \verb+HUB_INFO+ instruction.

The VIR is used for both hub instructions and user instructions. 
The VIR width (also known as the USER1 DR length) depends on the
width of the hub instructions and the width of the user instructions.
The width of the hub instructions depends on the number of nodes in the
design. The width of the user instructions depends on
the number of nodes in the design, and the {\em maximum}
width of the user-defined field of a user instruction, i.e., 
the maximum value of the \verb+SLD_IR_WIDTH+ parameter used on the
\verb+sld_virtual_jtag+ instances in a design, and the
maximum width of the instruction field required by any
other SLD JTAG components in the design, eg., SignalTap II logic
analyzer instances and SOPC Builder components such 
as the JTAG-to-Avalon Bridge and JTAG UART.

The width of the VIR $n$-bit hub/node address field shown in
Figure~\ref{fig:vir_encoding} is
%
\begin{equation}
n = \mbox{ceil}\left\{\log_2(N+1)\right\}
\label{eq:vir_width_n}
\end{equation}
%
where $N$ is the number of nodes in the design. The plus one
in this equation accounts for the fact that the hub uses an 
address (address zero). 

The width of the VIR $m$-bit instruction value field shown
in Figure~\ref{fig:vir_encoding} is
%
\begin{equation}
m = \mbox{max}(n+3, \mbox{\tt MAX\_SLD\_IR\_WIDTH})
\label{eq:vir_width_m}
\end{equation}
%
where $n+3$ is the field width required to encode the
VIR capture instruction, and \verb+MAX_SLD_IR_WIDTH+ is the
field width required to encode the maximum width of 
instructions used by the SLD components in the design
(instructions with less than $m$-bits are padded with
zeros before being issued).

The width $m$ can be read from the hardware using
the VIR \verb+HUB_INFO+ instruction. However, encoding the 
 \verb+HUB_INFO+ instruction to read the width $m$ requires
knowledge of $m$! This conflict 
is resolved by the fact that the encoding for the hub address
and the \verb+HUB_INFO+ instruction is all zeros, so the VIR
register can be loaded with zeros up to the maximum possible 
VIR width. The maximum total VIR width ($m+n$) can be determined as
follows;
%
\begin{itemize}
\item The maximum number of nodes is 255, i.e., $n = 8$.
\item The maximum user-defined \verb+SLD_IR_WIDTH+ is
32-bits, i.e., $m = 32$.
\end{itemize}
%
hence the maximum VIR width (for a design containing only
\verb+sld_virtual_jtag+ components) is 40-bits. Altera
recommends shifting 64-bits of zeros 
(pA-2~\cite{Altera_Virtual_JTAG_2008}).

Once the \verb+HUB_INFO+ instruction has been loaded into
the VIR, the 32-bit {\em Hub IP Configuration Register}
can be shifted out by performing eight 4-bit (nibble) DR-shift
operations. Figure~\ref{fig:hub_info} shows the fields within
this register. The Virtual JTAG User's Guide
\textcolor{red}{incorrectly} defines the first field as the
total VIR register width \verb+SUM(m,n)+, whereas this field
actually returns the VIR register $m$-width. The VIR field
width $n$ can be determined from the \verb+HUB_INFO+
{\em number of nodes} field, and the width $m$ is read
directly. From that point on, VIR instructions (including 
\verb+HUB_INFO+) can be encoded and issued with the correct 
number of bits. The \verb+HUB_INFO+ {\em number of nodes}
field indicates how many 32-bit \verb+SLD_NODE_INFO+ register
values can be read following the 32-bit \verb+HUB_INFO+
register. These 32-bit registers are read using the same
4-bit (nibble) DR-shift sequence as the \verb+HUB_INFO+
register. Figure~\ref{fig:node_info}
shows the register format and shows some example node
register values.

\begin{figure}[p]
\begin{center}
\includegraphics[width=0.95\textwidth]
{figures/jtag_vir_encoding.pdf}
\end{center}
\caption{JTAG Virtual Instruction Register (VIR) Encoding.}
\label{fig:vir_encoding}
\end{figure}

\begin{figure}[p]
\begin{center}
\includegraphics[width=0.95\textwidth]
{figures/jtag_hub_info_register.pdf}
\end{center}
\caption{JTAG Hub IP Configuration Register Format.}
\label{fig:hub_info}
\end{figure}

\setlength{\unitlength}{1mm}
\begin{figure}[p]
  \begin{picture}(160,80)(0,0)
    \put(2,5){
    \includegraphics[width=0.95\textwidth]
    {figures/jtag_node_info_register.pdf}}
    \put(73,58){(a)}
    \put(73,0){(b)}
  \end{picture}
  \caption{JTAG Node Information; (a) Register Format and
  (b) Example node information (for single node instances).}
  \label{fig:node_info}
\end{figure}

\clearpage
The equations for the Virtual Instruction Register (VIR) width
parameters $n$ and $m$, given in~\eqref{eq:vir_width_n} 
and~\eqref{eq:vir_width_m}, were confirmed using a DE2 board
design that instantiates multiple \verb+sld_virtual_jtag+
components. The design, \verb+sld_vjtag_multiple+, is 
parameterized with two generics;
%
\begin{itemize}
\item \verb+INSTANCES+

      The number of \verb+sld_virtual_jtag+ instances.

\item \verb+SLD_IR_WIDTH+

      The width of the user-defined component of the VIR register\newline
      (with all instances having the same width).
\end{itemize}
%
Table~\ref{tab:vir_width_exploration} shows the results of multiple
synthesis runs; the table contains the value calculated for $n$, and the 
two possible values for the total VIR register width, $(n+m)$. 
%
The table can be used to predict the VIR instruction width to
expect for a design containing \verb+sld_virtual_jtag+
components. The predicted width is determined by first selecting
the row consistent with the number of nodes in the design,
and then the column consistent with the (maximum) \verb+SLD_IR_WIDTH+
used within the design. If the text entry at the intersection of the
row and column values is black, then the VIR register width is 
dominated by the capture instruction, so the width is given by the
blue capture instruction width. If however, the entry text is
red, then the VIR register width is dominated by the user
instruction width and is given by the red text.

The VIR width parameters for a design are reported in multiple locations
in Quartus II;
%
\begin{itemize}
\item The parameters for the \verb+sld_virtual_jtag+
component instances are reported in the Virtual JTAG settings 
compilation report; {\em Analysis \& Synthesis}, {\em Debug Settings Summary},
{\em Virtual JTAG Settings}. 
%
\begin{itemize}
\item The USER1 DR length in this report is the total VIR width ($m+n$).
\item The capture instruction encodings for each instance are reported.
\item The address value reported for each instance are essentially a user
instruction with the user instruction value (the \verb+SLD_IR_WIDTH+ bits)
set to zero.
\end{itemize}
%
\item The SLD hub parameters can be obtained by hovering the mouse over
the \verb+sld_hub+ component in the Quartus II hierarchy window after
place-and-route of a design.
%
\begin{itemize}
\item \verb+n_nodes+ is the number of nodes in the design ($N$)
\item \verb+n_sel_bits+ is the number of select (address) bits ($n$)
\item \verb+n_node_ir_bits+ is the numbers of bits in the instruction field ($m$)
\item \verb+node_info+ is a binary value equivalent to the concatenation of
the 32-bit \verb+SLD_NODE_INFO+ values for each node in the design.
\end{itemize}
%
\item SLD component parameters are reported in the \verb+.jdi+ file for a 
project, eg. \verb+de2.jdi+.
%
\begin{itemize}
\item The file has the 32-bit \verb+SLD_NODE_INFO+ values in hexadecimal.
\item The file contains node details for all of the SLD components in a
design. For example, the DE2 board design \verb+sld_vjtag_custom+ can be
configured to contain an
\verb+sld_virtual_jtag+ component, a custom 
\verb+sld_virtual_jtag_basic+ component, a JTAG-to-Avalon bridge,
and a SignalTap II logic analyzer instance. The Quartus compilation reports
contain scattered details on each of these components. The \verb+de2.jdi+
file contains the component instance names, and JTAG identification details.
\end{itemize}
%
\end{itemize}
%
Figure~\ref{fig:sld_vjtag_to_gpio_settings} shows the Virtual JTAG
settings report and the SLD hub tooltip for the \verb+sld_vjtag_to_gpio+ 
project, while Figure~\ref{fig:sld_vjtag_multiple_settings}
shows the report for the \verb+sld_vjtag_multiple+ project.


% -----------------------------------------------------------------
% Hub parameters
% -----------------------------------------------------------------
%
\begin{landscape}
\begin{table}
\caption{Virtual Instruction Register (VIR) width parameter exploration.}
\label{tab:vir_width_exploration}
\begin{center}
\begin{tabular}{|c|c||c|c|c|c|c|c|c|c|c|c|c|c|c|}
\hline
            & VIR hub/node& \multicolumn{13}{l|}{VIR register width required for;}\\
\cline{3-15}
Number of   & address     & Capture     & \multicolumn{12}{l|}{User instruction ($n+m$)}\\
nodes, $N$  & width, $n$  & $(2n+3)$    & \multicolumn{12}{l|}{for  $m$ = {\tt SLD\_IR\_WIDTH} values of;}\\
\cline{4-15}
            &             &             & 1 & 2 & 3 & 4 & 5 & 6 & 7 & 8 & 9 & 10 & 11 & 12\\
\hline\hline
&&&&&&&&&&&&&&\\
   1    &  1 & \textcolor{blue}{ 5} &  2 &  3 &  4 &  \textcolor{red}{5} &  \textcolor{red}{6} &  \textcolor{red}{7} &  \textcolor{red}{8} &  \textcolor{red}{9} & \textcolor{red}{10} & \textcolor{red}{11} & \textcolor{red}{12} & \textcolor{red}{13}\\
  2-3   &  2 & \textcolor{blue}{ 7} &  3 &  4 &  5 &  6  &  \textcolor{red}{7} &  \textcolor{red}{8} &  \textcolor{red}{9} & \textcolor{red}{10} & \textcolor{red}{11} & \textcolor{red}{12} & \textcolor{red}{13} & \textcolor{red}{14}\\
  4-7   &  3 & \textcolor{blue}{ 9} &  4 &  5 &  6 &  7  &  8 &  \textcolor{red}{9} & \textcolor{red}{10} & \textcolor{red}{11} & \textcolor{red}{12} & \textcolor{red}{13} & \textcolor{red}{14} & \textcolor{red}{15}\\
  8-15  &  4 & \textcolor{blue}{11} &  5 &  6 &  7 &  8  &  9 & 10 & \textcolor{red}{11} & \textcolor{red}{12} & \textcolor{red}{13} & \textcolor{red}{14} & \textcolor{red}{15} & \textcolor{red}{16}\\
 16-31  &  5 & \textcolor{blue}{13} &  6 &  7 &  8 &  9  & 10 & 11 & 12 & \textcolor{red}{13} & \textcolor{red}{14} & \textcolor{red}{15} & \textcolor{red}{16} & \textcolor{red}{17}\\
 32-63  &  6 & \textcolor{blue}{15} &  7 &  8 &  9 & 10  & 11 & 12 & 13 & 14 & \textcolor{red}{15} & \textcolor{red}{16} & \textcolor{red}{17} & \textcolor{red}{18}\\
 64-127 &  7 & \textcolor{blue}{17} &  8 &  9 & 10 & 11  & 12 & 13 & 14 & 15 & 16 & \textcolor{red}{17} & \textcolor{red}{18} & \textcolor{red}{19}\\
128-255 &  8 & \textcolor{blue}{19} &  9 & 10 & 11 & 12  & 13 & 14 & 15 & 16 & 17 & 18 & \textcolor{red}{19} & \textcolor{red}{20}\\
&&&&&&&&&&&&&&\\
\hline
\end{tabular}
\end{center}
\end{table}
\end{landscape}
% -----------------------------------------------------------------

% -----------------------------------------------------------------
% Quartus Virtual JTAG report and tooltip for sld_vjtag_to_gpio
% -----------------------------------------------------------------
%
% Ctrl-Shift-Alt PrintScreen to get the images, then use Paint to
% save to PNG. For the tooltip, cut-and-paste into a new document
% in paint. Before pasting, reduce the image size to something
% smaller than the pasted bitmap. Do not use Corel PhotoPaint
% as the .PNGs it produces crashes PCTex.
%
\setlength{\unitlength}{1mm}
\begin{figure}
  \begin{picture}(160,160)(0,0)
    \put(2,65){
    \includegraphics[width=0.95\textwidth]
    {figures/sld_vjtag_to_gpio_vjtag_settings.png}}
    \put(40,5){
    \includegraphics[width=0.45\textwidth]
    {figures/sld_vjtag_to_gpio_hub_tooltip.png}}
    \put(73,60){(a)}
    \put(73,0){(b)}
  \end{picture}
  \caption{Quartus II Virtual JTAG Node Information for a single instance design;
  (a) Virtual JTAG settings compilation report, and
  (b) SLD hub tooltip information.}
  \label{fig:sld_vjtag_to_gpio_settings}
\end{figure}
% -----------------------------------------------------------------

% -----------------------------------------------------------------
% Quartus Virtual JTAG report and tooltip for sld_vjtag_to_multiple
% -----------------------------------------------------------------
%
\setlength{\unitlength}{1mm}
\begin{figure}
  \begin{picture}(160,160)(0,0)
    \put(2,65){
    \includegraphics[width=0.95\textwidth]
    {figures/sld_vjtag_multiple_vjtag_settings.png}}
    \put(25,5){
    \includegraphics[width=0.65\textwidth]
    {figures/sld_vjtag_multiple_hub_tooltip.png}}
    \put(73,60){(a)}
    \put(73,0){(b)}
  \end{picture}
  \caption{Quartus II Virtual JTAG Node Information for a multiple instance design;
  (a) Virtual JTAG settings compilation report, and
  (b) SLD hub tooltip information.}
  \label{fig:sld_vjtag_multiple_settings}
\end{figure}
% -----------------------------------------------------------------

\clearpage
The Tcl Virtual JTAG procedures use the instance index to access 
(address) each component instance; the reports in 
Figures~\ref{fig:sld_vjtag_to_gpio_settings} 
and~\ref{fig:sld_vjtag_multiple_settings} shows the instance
index for each \verb+sld_virtual_jtag+ component in the
design. The reports also show the instance capture instruction,
address, and USER1 DR length (VIR width).

Figure~\ref{fig:vir_encoding} shows the encoding for Virtual 
instructions. Table~\ref{tab:vir_instruction_examples} shows
example encodings. The \verb+SLD_IR_WIDTH+ value selected for 
these examples was such that the VIR width would be determined
by the capture instruction width, i.e., in 
Table~\ref{tab:vir_width_exploration} the blue values set the width,
eg., \verb+SLD_IR_WIDTH+ values of 1, 2, 3, or 4
can be used to generate the example encodings in
Table~\ref{tab:vir_instruction_examples}.
The hexadecimal values for the capture instruction and 
node address reported in Figures~\ref{fig:sld_vjtag_to_gpio_settings} 
and~\ref{fig:sld_vjtag_multiple_settings} correspond to
the $N=1$ and $N=8$ values in Table~\ref{tab:vir_instruction_examples}.

The examples in Table~\ref{tab:vir_instruction_examples} show the 
encoding for the capture instruction and the user address.
The capture instruction for each node consists of 
the $n$-bit hub address (zero), followed by the $n$-bit node
address, and then the 3-bit capture code 011b. Since the
VIR width for these examples was determined by the capture
instruction width, no zero padding is required
(see Figure~\ref{fig:vir_encoding}). 
The user address in Table~\ref{tab:vir_instruction_examples} 
consists of the $n$-bit node address, followed by zeros. A practical
use of this address is in the formation of user instructions;
the first $n$-bits is the node address, and the last \verb+SLD_IR_WIDTH+
bits encode the user instruction, with any unused bits between
the user instruction and node address being set to zero.

\begin{table}[t]
\caption{{\tt sld\_vjtag\_to\_gpio} Tcl procedures.}
\label{tab:sld_vjtag_to_gpio_cmds}
\begin{center}
\begin{tabular}{|l|p{10cm}|}
\hline
Command & Description\\
\hline\hline
&\\
\verb+jtag_open+  & Access JTAG on the DE2 board.\\
\verb+jtag_close+ & End JTAG access.\\
&\\
\verb+read_idcode+   & Read the Cyclone II JTAG IDCODE, i.e., 020B40DDh.\\
\verb+read_usercode+ & Read the user-defined JTAG USERCODE, eg., 12345678h.\\
\verb+pulse_nconfig+ & Reset the FPGA configuration.\\
&\\
\verb+print_hub_info+  & Print the Virtual JTAG \verb+HUB_INFO+.\\
\verb+print_node_info+ & Print the Virtual JTAG \verb+SLD_NODE_INFO+.\\
&\\
\verb+jtag_vir+ & Virtual IR-shift.\\
\verb+jtag_vdr+ & Virtual DR-shift.\\
&\\
\hline
\end{tabular}
\end{center}
\end{table}

The JTAG properties of a design can be queried using Tcl procedures.
Table~\ref{tab:sld_vjtag_to_gpio_cmds} shows the Tcl procedures implemented
in \verb+sld_vjtag_to_gpio/scripts/vjtag_cmds.tcl+.
The JTAG \verb+HUB_INFO+ and \verb+SLD_NODE_INFO+ commands were
issued to the designs shown in Tables~\ref{tab:vir_width_exploration} 
and~\ref{tab:vir_instruction_examples} to confirm the table
values.

% -----------------------------------------------------------------
% VIR encoding examples
% -----------------------------------------------------------------
%
\begin{landscape}
\begin{table}
\caption{VIR instruction encoding examples (for interpreting Quartus reported values).}
\label{tab:vir_instruction_examples}
\begin{center}
\begin{tabular}{|c|c|c||c|r|r|r|r|}
\hline
Number    & VIR hub/node & VIR     & Node       & \multicolumn{2}{c|}{Capture}     & \multicolumn{2}{c|}{Node (instance)}\\
of nodes, & address      & width,  & (instance) & \multicolumn{2}{c|}{Instruction} & \multicolumn{2}{c|}{Address} \\
\cline{5-8}
$N$       & width, $n$   & $(m+n)$ & Number     & \multicolumn{1}{c|}{(binary)} & \multicolumn{1}{c|}{(hex)} & \multicolumn{1}{c|}{(binary)} & \multicolumn{1}{c|}{(hex)} \\
\hline\hline
&&&&&&&\\
1 & 1 &  5 & 1 &       \verb+0_1_011b+ & \verb+0Bh+ &        \verb+1_0_000b+ & \verb+10h+\\
&&&&&&&\\
2 & 2 &  7 & 1 &     \verb+00_01_011b+ & \verb+0Bh+ &      \verb+01_00_000b+ & \verb+20h+\\
  &   &    & 2 &     \verb+00_10_011b+ & \verb+13h+ &      \verb+10_00_000b+ & \verb+40h+\\
&&&&&&&\\
4 & 3 &  9 & 1 &   \verb+000_001_011b+ & \verb+00Bh+ &   \verb+001_000_000b+ & \verb+040h+\\
  &   &    & 2 &   \verb+000_010_011b+ & \verb+013h+ &   \verb+010_000_000b+ & \verb+080h+\\
  &   &    & 3 &   \verb+000_011_011b+ & \verb+01Bh+ &   \verb+011_000_000b+ & \verb+0C0h+\\
  &   &    & 4 &   \verb+000_100_011b+ & \verb+023h+ &   \verb+100_000_000b+ & \verb+100h+\\
&&&&&&&\\
8 & 4 & 11 & 1 & \verb+0000_0001_011b+ & \verb+00Bh+ & \verb+0001_0000_000b+ & \verb+080h+\\
  &   &    & 2 & \verb+0000_0010_011b+ & \verb+013h+ & \verb+0010_0000_000b+ & \verb+100h+\\
  &   &    & 3 & \verb+0000_0011_011b+ & \verb+01Bh+ & \verb+0011_0000_000b+ & \verb+180h+\\
  &   &    & 4 & \verb+0000_0100_011b+ & \verb+023h+ & \verb+0100_0000_000b+ & \verb+200h+\\
  &   &    & 5 & \verb+0000_0101_011b+ & \verb+02Bh+ & \verb+0101_0000_000b+ & \verb+280h+\\
  &   &    & 6 & \verb+0000_0110_011b+ & \verb+033h+ & \verb+0110_0000_000b+ & \verb+300h+\\
  &   &    & 7 & \verb+0000_0111_011b+ & \verb+03Bh+ & \verb+0111_0000_000b+ & \verb+380h+\\
  &   &    & 8 & \verb+0000_1000_011b+ & \verb+043h+ & \verb+1000_0000_000b+ & \verb+400h+\\
&&&&&&&\\
\hline
\end{tabular}
\end{center}
\end{table}
\end{landscape}
% -----------------------------------------------------------------

Appendix A of the Virtual JTAG User Guide~\cite{Altera_Virtual_JTAG_2008}
\textcolor{red}{incorrectly} states that that \verb+HUB_INFO+ command
returns the total VIR width. Figure~\ref{fig:hub_info} shows the
instruction {\em actually} returns the VIR $m$-width. The error in the documentation
was confirmed in hardware using the \verb+HUB_INFO+ command.
For example, the \verb+sld_vjtag_multiple+ design with 4 instances
and an \verb+SLD_IR_WIDTH+ of 8, has \verb+HUB_INFO+;
%
\begin{verbatim}
tcl> print_hub_info
         Hub info: 0x08206E08
      VIR m-width: 8
  Manufacturer ID: 0x6E
  Number of nodes: 4
       IP Version: 1
\end{verbatim}
%
Table~\ref{tab:vir_width_exploration} indicates that the total VIR width
is determined by the user instruction and is 11-bits, and $n=3$, so
$m=11-3=8$, and this is what the \verb+HUB_INFO+
instruction returns. As another example, a design with 16 instances
and an \verb+SLD_IR_WIDTH+ of 5, has \verb+HUB_INFO+;
%
\begin{verbatim}
tcl> print_hub_info
         Hub info: 0x08806E08
      VIR m-width: 8
  Manufacturer ID: 0x6E
  Number of nodes: 16
       IP Version: 1
\end{verbatim}
%
Table~\ref{tab:vir_width_exploration} indicates that the total VIR width
is determined by the capture instruction and is 13-bits, and $n=5$, so 
$m=13-5=8$, and again this is what the
\verb+HUB_INFO+ instruction returns. 

\vskip20mm
% -----------------------------------------------------------------
\subsection{Multiple SLD debug component testing}
% -----------------------------------------------------------------
\label{sec:multiple_sld_tests}

This document has described the Virtual JTAG component
\verb+sld_virtual_jtag+ instruction encoding properties. 
The JTAG properties of other SLD components are not well documented
by Altera. To investigate the effect other SLD components have
on VIR encoding, the design \verb+sld_vjtag_custom+ was
created. The design can be configured to contain multiple SLD
components;
%
\begin{itemize}
\item One or two instances of \verb+sld_virtual_jtag+.
\item One or two instances of \verb+sld_virtual_jtag_basic+.
\item An instance of \verb+sld_virtual_jtag+ and \verb+sld_virtual_jtag_basic+.
\item An SOPC system containing a JTAG-to-Avalon bridge.
\item A SignalTap II logic analyzer instance.
\end{itemize}
%

\clearpage
\begin{table}[t]
\caption{Multiple SLD component VIR encoding results.}
\label{tab:sld_vjtag_custom}
\begin{center}
\begin{tabular}{|c||c|c||c|c|c|c|}
\hline
Example & SignalTap & SOPC    & Number of  & \multicolumn{3}{c|}{Virtual Instruction Register}\\
\cline{5-7}
Number  & Enabled   & Enabled & Nodes, $N$ & $n$ & $m$ & $(m+n)$\\
\hline\hline
&&&&&&\\
1 or 2     & $\times$   & $\times$   & 1 & 1 & 4 & 5\\
           & $\times$   & \checkmark & 2 & 2 & 5 & 7\\
           & \checkmark & $\times$   & 2 & 2 & 8 & 10\\
           & \checkmark & \checkmark & 3 & 2 & 8 & 10\\
&&&&&&\\
3, 4, or 5 & $\times$   & $\times$   & 2 & 2 & 5 & 7\\
           & $\times$   & \checkmark & 3 & 2 & 5 & 7\\
           & \checkmark & $\times$   & 3 & 2 & 8 & 10\\
           & \checkmark & \checkmark & 4 & 2 & 8 & 11\\
&&&&&&\\
\hline
\end{tabular}
\end{center}
\end{table}

\noindent The component \verb+sld_virtual_jtag_basic+ is the undocumented
component used to implement the \verb+sld_virtual_jtag+ component
(it can be seen in the hierarchy display in 
Figure~\ref{fig:sld_vjtag_to_gpio_settings}) and is the basis
of the JTAG PHY used in the JTAG-to-Avalon bridges. 
The \verb+sld_virtual_jtag_basic+ component allows you to
specify a manufacturer ID code, a device type ID code,
and a version number. This information can then be reported
using the \verb+SLD_NODE_INFO+ instruction. Unfortunately, once
you change the codes from the Altera values, the Virtual JTAG
Tcl procedures no longer work, i.e., you can no longer use
\verb+device_virtual_ir_shift+ and \verb+device_virtual_dr_shift+,
but must construct commands using the JTAG Tcl instructions
\verb+device_ir_shift+ and \verb+device_dr_shift+, which means
that user Tcl code has to deal with low-level VIR encoding.
Section~\ref{sec:altera_vjtag} describes a general-purpose
component based on the \verb+sld_virtual_jtag+ component that
provides identification registers, while allowing the use of
the Virtual JTAG Tcl procedures.

Table~\ref{tab:sld_vjtag_custom} shows synthesis results from the
\verb+sld_vjtag_custom+ design. The results are displayed in two
blocks; when the \verb+sld_vjtag_custom+ generic \verb+EXAMPLE+
is set to 1 or 2, the design contains a single \verb+sld_virtual_jtag+
or \verb+sld_virtual_jtag_basic+ component, and when the generic is
set to 3, 4, or 5, the design contains two components. The four
results for each of the two blocks indicate the change in VIR
parameters as either the SOPC system or SignalTap II instance is,
or both are, enabled in the design. The SOPC system contains a
JTAG-to-Avalon bridge. The bridge is implemented using an 
\verb+sld_virtual_jtag_basic+ component with an \verb+SLD_IR_WIDTH+
of 2; which is small enough that its instruction width is
capture instruction width dominated. Table~\ref{tab:sld_vjtag_custom} shows 
that when the design contains the Virtual JTAG instance(s)
and the SOPC system, the VIR width changes to accommodate the
change in node address bits.
When the SignalTap II instance is added, the VIR width increases
to accommodate the (undocumented) instruction width required
by the SignalTap II instance, which in this case is an effective
\verb+SLD_IR_WIDTH+ of 8-bits.

Tcl scripts that access Virtual JTAG components must interrogate
the hardware before issuing instructions, since the presence of
a SignalTap II logic analyzer instance will affect the instruction
encoding, and analyzer instances can be added to a design independently
of the design source, eg., VHDL. This is a key argument for designing
JTAG components around the \verb+sld_virtual_jtag+ component,
so that the Virtual JTAG Tcl commands can be used
(since they take care of hub interrogation).

\clearpage
% =================================================================
\section{Bus Functional Model (BFM) Simulation}
% =================================================================
\label{sec:simulation_model}

This section describes the design of a functionally correct simulation
model, i.e., the model generates transaction waveforms that
match logic analyzer traces shown in 
Sections~\ref{sec:external_logic_analyzer_traces}
and~\ref{sec:signaltap_logic_analyzer_traces},
with the exception of not reproducing the logical error present
in the Altera IP core, shown in Figures~\ref{fig:vji_shift_dr_hw_a},
\ref{fig:vji_shift_dr_hw_b} and~\ref{fig:vji_shift_dr_stp_a}.
The JTAG simulation infrastructure consists of;
%
\begin{itemize}
\item A bus functional model (BFM). 

The BFM consists of a VHDL server component, and a VHDL package
containing client/server communications procedures. 

Testbenches constructed using the BFM consist of a testcase generator,
the BFM server, and the device, or devices, under test (a design containing
one or more of the Virtual JTAG simulation models). The testcase generator
(the client) communicates with the BFM server using VHDL procedures 
analogous to the Virtual JTAG Tcl procedures. Procedure parameters are
passed between the client and server using a VHDL resolved signal
(read the BFM package source for details). The server receives client
requests and generates JTAG transactions on the server output signals
TCK, TMS, and TDI. The server outputs route to the inputs on each Virtual
JTAG model instance. The Virtual JTAG model TDO outputs route back
to the BFM server. The BFM server selects the appropriate TDO input 
based on the instance index indicated by the client (analogous to
the way the instance index argument is used by the Tcl procedures).

The BFM server is analogous to the Altera \verb+sld_hub+.

\item A Virtual JTAG model.

The top-level component \verb+altera_sld_virtual_jtag+ can be used to
replace the Altera Virtual JTAG component. The new component
contains {\em external} JTAG ports that are used only during simulation
(to connect to the BFM server).
The additional port definitions are contained within an Altera synthesis
directive\footnote{{\tt altera translate\_off} and 
{\tt altera translate\_on}.}, so that during synthesis the ports are 
not interpreted by Quartus.
Within the component, the design instantiates either the
simulation component \verb+altera_sld_virtual_jtag_model+ or the
synthesis component \verb+sld_virtual_jtag+. Altera synthesis directives
are again used\footnote{{\tt synthesis read\_comments\_as\_HDL on} and
{\tt synthesis read\_comments\_as\_HDL off}.}, so that Quartus only sees
the synthesis model.
\end{itemize}


The \verb+altera_sld_virtual_jtag_model+ component contains an
\verb+altera_jtag_tap+ component that tracks the state of the
JTAG transaction, and an instruction register for determining
when the USER1 and USER0 instructions have been issued.
The component {\em combinatorially} generates the virtual
JTAG one-hot outputs using the instruction encoding and
the JTAG state. The logic analyzer traces show that the
VIR register does not change when a capture instruction is
issued, only when the user instruction is issued, to the
model reproduces this sequence. 

The simulation model architecture does not match the 
hardware architecture; a simulation design containing multiple
instances of the simulation model will contain redundant copies
of the 10-bit JTAG instruction register for USER0/USER1
detection. However, containing this repeated logic within the
simulation model makes the simulation infrastructure simpler,
as the connection between the BFM server and the models
reduces to the JTAG signals, rather than the JTAG signals, 
USER0 and USER1 decode signals, and Virtual Instruction
Register buses (one for each instance). The compilation reports
for the single and multiple instance designs in
Figures~\ref{fig:sld_vjtag_to_gpio_settings} 
and~\ref{fig:sld_vjtag_multiple_settings} show that
the \verb+sld_virtual_jtag+ components consume few or zero 
logic resources, whereas the \verb+sld_hub+ consumes an 
increasing amount of logic (96 logic cells, then 298 logic cells).
This implies that the \verb+sld_hub+ is where all the Virtual 
JTAG logic resides, and that the \verb+sld_virtual_jtag+ component is
simply the user-visible interface to the hub logic.

The simulation model does not implement the \verb+HUB_INFO+ and
\verb+SLD_NODE_INFO+ registers. The logic is not difficult
to model, however, the testbench needs the generics that these
instructions encode, so its just as easy for the client/server 
procedures to use the generics directly (otherwise the client
procedures would have to first interrogate the hub for the 
information). The simulation models use generics to construct
VIR encode and decode logic that matches the hardware 
implementation.

Figure~\ref{fig:altera_sld_vjtag_shift_ir_tb} shows the simulation
model Virtual IR-shift sequence. Comparison of the figure to
the logic analyzer traces in Figures~\ref{fig:vji_shift_ir_hw_a}
and~\ref{fig:vji_shift_ir_stp_a}, shows that the simulation
model generates the correct JTAG sequence. Contrast this to the
incorrect simulation results produced by the Altera model in
Figure~\ref{fig:vji_shift_ir_tb_a}.

Figure~\ref{fig:altera_sld_vjtag_shift_dr_tb} shows the simulation
model Virtual DR-shift sequence. Comparison of the figure to
the logic analyzer traces in Figures~\ref{fig:vji_shift_dr_hw_a}
and~\ref{fig:vji_shift_dr_stp_a}, shows that the simulation
model generates the correct JTAG sequence. Contrast this to the
incorrect simulation results produced by the Altera model in
Figure~\ref{fig:vji_shift_dr_tb_a}.


\begin{figure}[p]
  \centering
  \includegraphics[width=\textwidth]
  {figures/altera_sld_vjtag_vir_tb.pdf}
  \caption{Virtual JTAG simulation; Virtual IR-shift sequence.}
  \label{fig:altera_sld_vjtag_shift_ir_tb}
\end{figure}

\begin{figure}[p]
  \centering
  \includegraphics[width=\textwidth]
  {figures/altera_sld_vjtag_vdr_tb.pdf}
  \caption{Virtual JTAG simulation; Virtual DR-shift sequence.}
  \label{fig:altera_sld_vjtag_shift_dr_tb}
\end{figure}

\clearpage
% =================================================================
\section{General-purpose identifiable Virtual JTAG component}
% =================================================================
\label{sec:altera_vjtag}

Custom components based on the \verb+sld_virtual_jtag+ component
require some form of identification to uniquely identify each
instance. The identification information can then be used by
Tcl scripts provided in a general-purpose Tcl package that
can be reused between designs. For example, consider a user-defined
JTAG-to-Avalon master bridge component. That component would
be supplied along with a Tcl package containing procedures for
performing Avalon bus read and write transactions.
The \verb+SLD_NODE_INFO+ command can be used to identify the
index of an \verb+sld_virtual_jtag+ instance within a design,
but it can not provide information on what that component
is used to implement. A design-specific Tcl script could be
written that provides Avalon read and write transaction to say
instance index zero, however, this does not allow for design reuse
(other than by copying the Tcl script). The issue of component
identification becomes more difficult when a design is created
with SOPC Builder, because when the \verb+sld_virtual_jtag+ components
use automatic instance indexing, the component indexes
change depending on the order of components in the generated
SOPC system (an ordering that the user can not easily control).

Altera gets around this identification issue by using the
\verb+sld_virtual_jtag_basic+ component to implement the
\verb+sld_virtual_jtag+ and JTAG-to-Avalon bridge components with
unique identification information. The Altera Tcl procedures
then check that identification information before generating
component-specific JTAG transactions. A custom user component
could be built using the \verb+sld_virtual_jtag_basic+ component,
however, as discussed in Section~\ref{sec:multiple_sld_tests},
if you change the  Virtual JTAG component identification code, 
the Virtual JTAG Tcl procedures no longer work. The Virtual
JTAG Tcl procedures hide the complexities of virtual instruction 
encoding from the user, and it would be useful to preserve this 
functionality (rather than re-implement it using the low-level JTAG
Tcl IR-shift and DR-shift procedures). Hence, a custom 
Virtual JTAG component should be implemented using
the \verb+sld_virtual_jtag+ component.

% -----------------------------------------------------------------
% altera_vjtag generics
% -----------------------------------------------------------------
%
\begin{table}[t]
\caption{General-purpose identifiable Virtual JTAG component, {\tt altera\_vjtag}, generics.}
\label{fig:altera_vjtag_generics}
\begin{center}
\begin{tabular}{|l|p{8cm}|}
\hline
Generic & Description\\
\hline\hline
\multicolumn{2}{|l|}{\bf User Interface Parameters:}\\
\hline
&\\
\verb+UIR_WIDTH+  & User instruction width.\\
&\\
\hline
\multicolumn{2}{|l|}{\bf Identification Parameters:}\\
\hline
&\\
\verb+VID+           & 16-bit Vendor ID (VID) code.\\
\verb+PID+           & 16-bit Product ID (PID) code.\\
\verb+VERSION+       & 16-bit Version.\\
\verb+NODE_INDEX+    & The node index (reported by \verb+SLD_NODE_INFO+).\\
&\\
\hline
\multicolumn{2}{|l|}{\bf Simulation Parameters:}\\
\hline
&\\
\verb+MAX_SLD_IR_WIDTH+ & Maximum value of \verb+SLD_IR_WIDTH+.\\
\verb+NUMBER_OF_NODES+  & Number of Virtual JTAG nodes.\\
&\\
\hline
\end{tabular}
\end{center}
\end{table}
% -----------------------------------------------------------------

% -----------------------------------------------------------------
% altera_vjtag block diagram
% -----------------------------------------------------------------
%
\begin{figure}[p]
  \centering
  \includegraphics[width=\textwidth]
  {figures/altera_vjtag_diagram.pdf}
  \caption{General-purpose identifiable Virtual JTAG component, {\tt altera\_vjtag}.
  The diagram shows a component instance with a user-defined instruction
  width, {\tt UIR\_WIDTH}, of 4. The internal implementation instantiates
  a Virtual JTAG component with an {\tt SLD\_IR\_WIDTH} of 5. The most
  significant bit of the instruction register, {\tt ir\_in[4]}, is defined
  to be the {\tt user} mode select bit; when {\tt user} is high the user 
  TDO, {\tt user\_tdo}, drives the JTAG TDO signal, but when 
  {\tt user} is low the identification TDO, {\tt id\_tdo}, drives the JTAG 
  TDO signal, with {\tt ir\_in[1:0]} selecting the identification register.
  The internal shift-register load and shift control signals only assert
  when the corresponding virtual state signal asserts and {\tt user} is low. 
  The external shift-register controls
  {\tt dr\_capture}, {\tt dr\_shift}, and {\tt dr\_update} only assert  
  when the corresponding virtual state signal asserts and {\tt user} is high.}
  \label{fig:altera_vjtag_diagram}
\end{figure}
% -----------------------------------------------------------------

Figure~\ref{fig:altera_vjtag_diagram} shows a block diagram of the
\verb+altera_vjtag+ component which implements a general-purpose 
identifiable Virtual JTAG component. The figure caption describes 
the internal implementation of the component (the VHDL source code
can be reviewed for more details). Table~\ref{fig:altera_vjtag_generics}
describes the generics used to configure the component.

Auto-detection of components based on the \verb+altera_vjtag+ component
proceeds as follows;
%
\begin{enumerate}
\item For each instance of the \verb+sld_virtual_jtag+ component identified
by \verb+SLD_NODE_INFO+, issue a virtual user instruction of zero.
This sets the \verb+ir_in[]+ port to all zeros, i.e., \verb+user+ low,
so that the identification registers are selected in the component JTAG chain.
\item Read the 16-bit Vendor ID using a 16-bit Virtual DR-shift.
\item If the Vendor ID code is recognized, then issue virtual user
instructions of 1, 2, and 3, and use 16-bit Virtual DR-shifts to read
the other identification registers.
\item Select the user registers by setting the \verb+user+ instruction bit 
high. The position of this bit is determined from the value read from
the \verb+SLD_IR_WIDTH+ identification register, i.e.,\newline
\verb+user = ir_in[SLD_IR_WIDTH-1]+.
\item Perform further device specific register accesses.
\item Enable the use of component-specific Tcl procedures with this
instance index.
\end{enumerate}
%
A key aspect of the identification encoding scheme is to use the
\verb+user+ low value to select the identification registers.
The position of the \verb+user+ bit within the instruction register
is not known until the identification registers are read.
The VIR $m$-width from the \verb+HUB_INFO+ instruction can not
be used for this purpose, since that value represent the
{\em maximum} value of \verb+SLD_IR_WIDTH+ used in all component
instances within a design, not the specific instance of the
\verb+altera_vjtag+ component.

Another subtle aspect of the \verb+altera_vjtag+ component is the
use of the virtual state signals; the use of the capture and shift
states is fairly obvious, but the use of \verb+virtual_state_e1dr+
to generate the update pulse is less so. The reason this signal
is used, rather than the \verb+virtual_state_udr+ signal, is that
\verb+virtual_state_e1dr+ is pulsed for one JTAG clock period 
following the DR-shift sequence, whereas \verb+virtual_state_udr+ 
goes high, and stays high because once the JTAG TAP controller
reaches that state, the external JTAG controller stops generating
the JTAG clock, eg., see the JTAG and Virtual JTAG state
assertions in Figure~\ref{fig:altera_sld_vjtag_shift_dr_tb}.

The testbench \verb+altera_vjtag_tb+ instantitates the testcase
\verb+altera_vjtag_tc+, the JTAG BFM server, and an instance
of the \verb+altera_vjtag+ component with its user instruction
register looped-back on itself, and user TDO fed by a signal
that toggles every falling edge of TCK.
The testcase generator demonstrates the component identification
sequence, and demonstrates the use of VHDL assertions for
verification. This testbench can be used
as the basis for any custom \verb+altera_vjtag+ component. 
Section~\ref{sec:examples} contains additional example applications
of the \verb+altera_vjtag+ component.


\vskip10mm
\noindent {\bf TODO:}
\begin{itemize}
\item Tcl package for the core commands for this component, eg.,
\verb+package require vjtag+. Create a version of \verb+print_identification+
that loops over all nodes within a design. Create an example test
design containing identifiable and non-identifiable \verb+sld_virtual_jtag+ components.
\item Get the \verb+altera_vjtag+-based JTAG-to-Avalon bridge working and documented.
\end{itemize}

\clearpage
% =================================================================
\section{Design Examples}
% =================================================================
\label{sec:examples}

This section contains synthesizeable Virtual JTAG example
designs.

Mention the DE2 board \verb+sld_virtual_jtag+ based designs,
and the DE2 \verb+altera_vjtag+ design.

Clock-domain crossing example.

Provide a couple of basic examples, and then example traces
from a custom JTAG-to-Avalon master. Then just refer the
reader to my Avalon document.

Tcl GUI to control the DE2 board? Maybe save that for use with
the JTAG-to-Avalon bridge.

How fast is TCK on the DE2 board? How about on the BeMicro?

Actually, what examples did I try on the BeMicro? Did I check
them into CVS? Didn't I use the VHDL keep signals attributes
in the example design?

External TCK/TMS/TDI/TDO signal capture - route to GPIO pins
on something (S4GXDK) and trace VIR sequences.


% -----------------------------------------------------------------
\subsection{User Guide Examples}
% -----------------------------------------------------------------

Review the user guide examples; DCFIFO and a read-only timestamp 
or ID - right? The example was lame in that you'd want that
register to be read via JTAG or onboard processor, so really
the interface should be an Avalon Master.

Create a simulation version of the examples? Did they include
clock domain synchronizers?

% -----------------------------------------------------------------
\subsection{LED control and switch status}
% -----------------------------------------------------------------

The DE2 project connects the Virtual instruction output port to
the red and green LEDs, and the Virtual instruction
input port to the switches. The LEDs on the board can be
illuminated, and the switch state read-back, via the sequence;
%
\begin{verbatim}
tcl> jtag_open
JTAG: USB-Blaster [USB-0], FPGA: @1: EP2C35 (0x020B40DD)
tcl> device_lock -timeout 10000
tcl> device_virtual_ir_shift -instance_index 0 -ir_value 7
5
tcl> device_unlock
tcl> jtag_close
\end{verbatim}
%
where since the Virtual instruction width on the JTAG instance
was set to 3-bits, three of the red and green LEDs are illuminated
by this instruction. The 3-bit switch state read-back was
5, i.e., two switches were on and one was off. The \verb+de2.vhd+
file contains a \verb+VIR_WIDTH+ parameter that can be set from
1 to 32-bits (per the limits on this value imposed by the
Virtual JTAG component). The LEDs and switches will be wired
according to the \verb+VIR_WIDTH+ (until all LEDs and switches
have been connected).

% -----------------------------------------------------------------
\subsection{Control and status register}
% -----------------------------------------------------------------

Simple JTAG component applications.

Using an \verb+sld_virtual_jtag+ component without additional
logic; connect \verb+ir_in+ to outputs, eg. LEDs, and \verb+ir_out+
to inputs, eg., switches. Or if you want write/read bits connect
\verb+ir_in+ bits to the LED, and connect it to \verb+ir_out+
for read-back.

\verb+de2/cyclone2/sld_vjtag_to_gpio/+ already shows an example
of this approach.

% -----------------------------------------------------------------
\subsection{Read/write registers block}
% -----------------------------------------------------------------

Draw a nice version of my read/write registers block figure.
Point out that the write-enable decoding disappears when
the registers become read-only, and that is how the 
\verb+altera_vjtag+ component identification registers
works - create a figure for the read-only registers.

SignalTap II trace of register accesses?

Timing analysis; really needs to use an FPGA clock. Eg.
synchronize the \verb+virtual_jtag_uir+ to the FPGA
clock via a pulse synchronizer, use that synchronizer
to enable an address register, and register the output
of the registers block. Then you can apply a timing
constraint on the FPGA clock domain and cut paths to
the JTAG interface.

% -----------------------------------------------------------------
\subsection{Clock and reset checking}
% -----------------------------------------------------------------

Connect the clock or reset for a design in the TDO path and
perform a Virtual DR-shift operation of say 32-bits. If
the value read back is all zeros or ones, then the signal
is at a static level, otherwise its toggling.

% -----------------------------------------------------------------
\subsection{Clock-domain crossing}
% -----------------------------------------------------------------


\verb+de2/altera_vjtag/+ just uses a single clock domain,
that of JTAG. Extend this example to be able to read the state
of the reset control (one of the keys) and to read the clock
(\verb+clk_50MHz+), or use my pulse-count logic, eg. toggle a signal
using JTAG, and see if the counter enabled by that signal
increments. If it does, and the count is fairly large
(consistent with the expected clock frequency) then there is a
clock present. Also try the simple 'read the clock value'
approach to detect if its toggling.

% -----------------------------------------------------------------
\subsection{Address and data bus interface}
% -----------------------------------------------------------------

\verb+vjtag_to_registers_bridge+ Like my VJTAG component.

Encode \verb+ADDR_WIDTH+, \verb+BYTEEN_WIDTH+,
\verb+DATA_WIDTH+ in some read-only registers, then provide a
control/status register (or at least locations for them for
the test examples), eg. a 32-bit read/write control register
and a 32-bit read-only status register. I'd then use the
same scheme for my Avalon bridge (but restrict bus parameter
values to those in the Avalon specification, eg. 8-bit increments).
The control register will need a bit to indicate whether the
transaction is a read or write. Add another control bit that
determines the addressing scheme; fixed address or automatic address
increment. The address control bit will allow you to write a start
address (which will load a counter), and then write data
to increasing addresses, and similar for reads (saves on
JTAG activity). I'm not sure if I have enough JTAG clocks
to make this work, so perhaps only implement it for the
bridge that uses two clocks. Now that I have a simulation
model, I can figure out whether this will work with just
the JTAG clock.

Refer to the Avalon bridges, but stick them in my Avalon document:
\verb+vjtag_to_avalon_mm_bridge+,
\verb+vjtag_to_avalon_st_bridge+,
JTAG-to-Avalon Bridges done right, i.e., you can simulate with them.
The Altera JTAG-to-Avalon bridge can only be simulated using
Verilog PLI and the system console, and that interface is
unsupported and barely documented (the Altera Wiki has instructions
on what to do).

% -----------------------------------------------------------------
\subsection{Dual-port RAM test}
% -----------------------------------------------------------------

Assuming the RAM clock is faster than the JTAG clock, then
the fact that when the RAM read data takes a couple of clocks
before it is valid is generally acceptable. Eg. RAM address,
byte-enable, asserted, and then by the time the read command
is decoded from the JTAG interface, the read data has been
valid for ages. However, the RAM clock could be made so slow
that this is violated. When accessing a device that can take
a while to response, some form of read-data-ready status
register bit is required.

\verb+de2/cyclone2/vjtag_ram_example/+

% -----------------------------------------------------------------
\subsection{Dual-clock FIFO test}
% -----------------------------------------------------------------

Same deal as with RAM, but the JTAG address space is slit to
decode to the FIFO input or output. The FIFO status signals
can be routed to a read-only status register location.
The FIFO can be operated in dual clock mode, and as long
as those clocks are faster then the JTAG clock, the JTAG
interface will work fine. You can write to the FIFO while
monitoring the used value and the FIFO status outputs, and
the drain the FIFO. Write a simulation and synthesis test.

\verb+de2/cyclone2/vjtag_fifo_example/+

\clearpage
% =================================================================
% Appendices
% =================================================================
\appendix
% -----------------------------------------------------------------
\section{Altera Tool Versions}
% -----------------------------------------------------------------

Quartus 9.1, Quartus 10.1, Modelsim SE 6.5b (full version),
Modelsim Altera-Edition (use both the versions expected to
be used with Quartus 9.1 and 10.1).

Tcl scripts create version specific build directories so
all these tools can be installed at the same time, and all
the build examples can exist at the same time.

% -----------------------------------------------------------------
\section{Source Code Description}
% -----------------------------------------------------------------
\label{app:source_code}

VHDL and Tcl source.

Describe the use of the synthesis scripts for building a 
board design.

Describe the use of a simulation script for setting up
procedures that run a simulation.

List the project names and folders - a table perhaps with
a description of each project?

Describe how to create a new DE2 project, by copying the
DE2 \verb+basic+ design.


%
\begin{itemize}
\item New project folders were created in the DE2 board
project directory, i.e.,

\begin{tabular}{ll}
\verb+$VHDL/boards/de2/cyclone2/sld_virtual_jtag/+ & Project folder\\
\verb+$VHDL/boards/de2/cyclone2/sld_virtual_jtag/src/+ & VHDL source\\
\verb+$VHDL/boards/de2/cyclone2/sld_virtual_jtag/test/+ & VHDL testbench\\
\verb+$VHDL/boards/de2/cyclone2/sld_virtual_jtag/scripts/+ & Scripts\\
\end{tabular}
%
\item The top-level VHDL file and synthesis
script were copied from the \verb+basic+ design, i.e.,

\verb+$VHDL/boards/de2/cyclone2/sld_virtual_jtag/src/de2.vhd+\\
\verb+$VHDL/boards/de2/cyclone2/sld_virtual_jtag/scripts/synth.tcl+.
\end{itemize}

The \verb+basic+ design contains a VHDL entity with
ports for all the pins used on the DE2 board, along with
a basic architecture body that connects switches to hexadecimal
displays and deasserts control signals to external devices.
%


% -----------------------------------------------------------------
\section{JTAG Timing Constraints (TimeQuest SDC)}
% -----------------------------------------------------------------

Check the Cyclone II and II databooks, I'm sure that JTAG signals
are supposed to be clocked like SPI, i.e., data changes on the
falling-edge of the clock, and is captured on the rising-edge.
Comment that the logic analyzer
traces show this is not the case. Screen shots showing the
SDC timing constraints for JTAG. What is the timing margin
on the signals? 

Create a generic JTAG script. How can the script determine if the
JTAG signals have been used? You should be able to look for signals
and then if they are found in the design, apply a constraint.

Point out that Altera's SDC examples for JTAG are not consistent
with the falling-edge clocking.

How about their STAPL documentation, how does that say the
JTAG signals should be clocked? Look at my Byteblaster and
COBRA example code. What did DaveM do in the CARMA code?


\clearpage
%------------------------------------------------------------------
% Do the bibliography
%------------------------------------------------------------------
%Note, you can't have spaces in the list of bibliography files
%
\bibliography{refs}
\bibliographystyle{plain}

%------------------------------------------------------------------
\end{document}






















































